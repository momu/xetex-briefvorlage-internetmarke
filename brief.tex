% !TeX spellcheck = de_DE
% !TeX encoding = utf8
% !TeX program = xelatex
%%
% (German) XeTex letter template based on the template 
% from Jan-Philip Gehrcke
% http://gehrcke.de -- jgehrcke@gmail.com -- November 2009
% 
% Source: http://gehrcke.de/2009/12/latex-briefvorlage/
%%

% DIV and BCOR settings result in smaller/wider side margins
% see scrguide @
% http://www.tex.ac.uk/tex-archive/macros/latex/contrib/koma-script/scrguide.pdf

% so ist der brief nicht strikt nach din 5008
% mit den werten für {geometry} wird er zu einem din 5008 typ b brief (der satzspiegel (ränder)) ändern sich
% das ist aber für private briefe nciht zwingend vorgeschrieben und sieht laut latexexperten auch
% typographisch falsch aus
% 

\documentclass[%
	%nonsymm,
 	fontsize=11pt,	%Schriftgroesse (12pt default)
 	paper=a4,	%Papierformat (A4 default)
 	%DIV=12,		%Seitenaufteilung (12 default -- siehe scrguide.pdf), hat einfluss auf die ränder
% 	BCOR=5mm,	%Zusaetzlicher Rand auf der Innenseite zur Bindekorrektur
	parskip=half*,  %Absatz statt Einzug
         %draft,        %Draftmodus zum Debugging
        %moretext,       %DINmtext statt DIN
        %DINfalse,       %Satzspiegel scrlttr2-Standard, aber nicht strikt DIN5008-konform!
%         nonsymm,        %historisch :)
%         nonexthead,     %leerer nexthead auf den Folgeseiten
	ngerman,        %Briefsprache Deutsch
% 	english,        %Briefsprache Englisch
	version=last,
	fromlogo,
	]{scrlttr2}
\LoadLetterOption{letter_options}
\usepackage{xltxtra}

\usepackage{polyglossia}
\usepackage{fontspec}

\setmainlanguage[latesthyphen=true,babelshorthands=true]{german}
\usepackage{adjustbox}
\usepackage[absolute]{textpos} 

\defaultfontfeatures{Mapping=tex-text}
%\setromanfont{Linux Libertine O}
%\setsansfont{Linux Biolinum O}
\setmainfont{Vollkorn}
% overall sans serif font
%\renewcommand{\familydefault}{\sfdefault}

%\enlargethispage{\baselineskip} % alles ein wenig quetschen, damit es auf eine seite passt

% enable to get din5008 outlines
%\input{din5008frame.tex}

\begin{document}

\setkomavar{fromlogo}{%
\setlength{\unitlength}{1mm}

\begin{picture}(0,0)
\put(21,-19){
	\adjincludegraphics[trim=27mm 245mm 125mm 37mm,clip]{internetmarke.pdf}}
\end{picture}
}
      
\setkomavar{subject}{Ihr Schreiben vom ...}
\setkomavar{date}{\today}
\setkomavar{place}{Stadt}
% German address
\begin{letter}{

Empfänger\\
Straße 1\\
12345 Stadt
}

\opening{Sehr geehrte Damen und Herren,}
Lorem ipsum dolor sit amet, consectetur adipiscing elit. Donec a diam lectus. Sed sit amet ipsum mauris. 
Maecenas congue ligula ac quam viverra nec consectetur ante hendrerit. Donec et mollis dolor. 
Praesent et diam eget libero egestas mattis sit amet vitae augue. 

Nam tincidunt congue enim, ut porta lorem lacinia consectetur. Donec ut libero sed arcu 
vehicula ultricies a non tortor. Lorem ipsum dolor sit amet, consectetur adipiscing elit. 
Aenean ut gravida lorem. Ut turpis felis, pulvinar a semper sed, adipiscing id dolor. Pellentesque auctor 
nisi id magna consequat sagittis. Curabitur dapibus enim sit amet elit pharetra tincidunt feugiat nisl imperdiet.
 Ut convallis libero in urna ultrices accumsan. Donec sed odio eros. Donec viverra mi quis quam pulvinar at 
 malesuada arcu rhoncus. Cum sociis natoque penatibus et magnis dis parturient montes, nascetur ridiculus mus. 
 In rutrum accumsan ultricies. Mauris vitae nisi at sem facilisis semper ac in est.


%\nopagebreak[3]
\closing{Mit freundlichen Grüßen}
%\hspace{-2cm} %hochschieben
%\includegraphics[width=0.3\textwidth]{../unterschrift.jpg}
%\hspace{2cm} % wieder runter

%\begin{textblock}{5}(1,11.6)      % zweite Zahl in der zweiten Klammer verändert die vertikale Position der Unterschrift 
%  \includegraphics[width=5.5cm]{../unterschrift.jpg} 
%\end{textblock} 

\setkomavar*{enclseparator}{Anlage}%
\encl{Anlage I\\Anlage II}%
%\cc{Medea Mittel, Willi Weg}
%\ps{Nicht persönlich nehmen!}
\end{letter}
\end{document}
